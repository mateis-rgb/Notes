\documentclass{article}

\usepackage[utf8]{inputenc}
\usepackage[T1]{fontenc}

\title{Base d'informatique - Fiche de révision}
\author{Matéis R.}
\date{Octobre 2023}

\begin{document}
	\maketitle

	\section{Conversion d'une base $b$ vers la base décimale}
		Pour convertir des nombres codés en base 2, 8, 16 et 20, la méthode est la même. On multiplie chacun des chiffres par la base $b$ élevée à la puissance de l'indice où se trouve le chiffre en question. Voici le tableau des conversions pour aller plus vite :  
		
		\begin{center}
			\begin{tabular}{ | c | c | c | c | c | c | c | c | c | c | c | c | }
  				\hline
					$n$ & 10 & 9 & 8 & 7 & 6 & 5 & 4 & 3 & 2 & 1 & 0 \\ 
 				\hline
  					 $2^n$ & 1024 & 512 & 256 & 128 & 64 & 32 & 16 & 8 & 4 & 2 & 1 \\ 
	  			\hline
					$8^n$ & & & & & & & 4096 & 512 & 64 & 8 & 1 \\
				\hline
					$10^n$ & & & & & & & & 1000 & 100 & 10 & 1 \\
				\hline
					$16^n$ & & & & & & & & 4096 & 256 & 16 & 1 \\
				\hline
					$20^n$ & & & & & & & & 8000 & 400 & 20 & 1 \\
				\hline
			\end{tabular}
		\end{center}
		
		Ainsi, on regarde dans le tableau les valeurs qui nous intéressent et on les additionne.
	
	\section{Addition en binaire}
		Voici les règles d'addition en binaire :
		
		\begin{center}
			$0_2 + 0_2 = 0_2$ \\
			$0_2 + 1_2 = 1_2$ \\
			$1_2 + 0_2 = 1_2$ \\
			$1_2 + 1_2 = 10_2$ \\
		\end{center}	
		Pour la dernière règle ($1_2 + 1_2 = 10_2$), on abaisse le 0 et on génère une retenue de 1.

	\section{Les entiers relatifs}
		On s'intéresse à présent à la représentation des entiers relatifs. Pour représenter les nombres négatifs, on utilise la notation dite binaire en complément à deux.
	
		\begin{center}
			\begin{tabular}{ | c | c | c | c | }
  				\hline
  					Nb bits & Valeur Minimum & Valeur Maximum & Nb Valeurs \\
				\hline
					8 & -128 & 127 & 256 \\
				\hline
					16 & $-32,768$ & $32,767$ & $65,536$ \\
				\hline
					32 & $-2,147,483,648$ & $2,147,483,647$ & $4,294,967,296$ \\
				\hline
			\end{tabular}
		\end{center}
		
		Si on veut convertir un nombre négatif, on doit :
		\begin{enumerate}
			\item On prend la valeur absolue du nombre.
			\item On la code en binaire sur le nombre choisi de bits.
			\item On utilise le complément à deux de la valeur codée en inversant les 0 et les 1.
			\item On ajoute 1 au résultat.
		\end{enumerate}
		
		Par exemple :

		\begin{center}
			$-18_{10}$ \\
			$18_{10}$ \\
			$0001,0010_2$ \\
			$1110,1101_{\bar2}$ \\
			$1110,1110_{\bar2}$ \\			
		\end{center}

		Ainsi, -18 s'écrit $1110,1110_{\bar2}$ en binaire.
	
	\section{Coder un réel flottant}
		Pour coder un réel flottant au format IEEE 754 en simple précision (sur 32 bits), on commence par s'intéresser au signe du réel. On dit alors que $S = 1$ si le réel est négatif et $S = 0$ sinon. Ensuite, on code la partie entière en valeur absolue, que l'on appelle la mantisse. Pour la partie décimale, on utilise des puissances de deux négatives (voici un tableau des puissances de deux négatives, ci-dessous). Plus simplement, pour coder en binaire la partie décimale, il faut multiplier la partie décimale par 2 jusqu'à obtenir 0. À chaque étape, on garde le chiffre le plus à gauche du résultat (c'est-à-dire la partie entière) de la multiplication, qui sera 1 ou 0. Ensuite, on réitère la multiplication sur la partie décimale du résultat de la multiplication, en supprimant le premier 1 s'il existe.
		
		\begin{center}
			\begin{tabular}{ | c | c | c | c | c | c | }
  				\hline
  					n & 0 & -1 & -2 & -3 & -4 \\
  				\hline
  					$2^n$ & 1 & $\frac{1}{2} = 0,5$ & $\frac{1}{4} = 0,25$ & $\frac{1}{8} = 0,125$ & $\frac{1}{16} = 0,0625$ \\
	  			\hline
			\end{tabular}
		\end{center}
	
		Une fois que l'on a converti toutes les parties du nombre réel en binaire, dans la partie entière (codée en binaire), on décale la virgule vers la gauche jusqu'au premier 1 qui compose la mantisse. Une fois cela fait, on obtient la mantisse tronquée. Maintenant qu'on a déplacé la virgule, on compte le nombre de rangs que l'on a déplacé. On ajoute à ce nombre 127 et le convertit en binaire. On obtient ainsi l'exposant décalé. 
		\\ \\
Après cela, on place le bit du signe ($S$), puis l'exposant décalé ($E_d$), enfin la mantisse tronquée ($M_t$). Pour terminer, on ajoute des 0 pour obtenir un total de 32 bits.		
		
		\begin{center}
			\begin{tabular}{ | l | c | c | c | }
				\hline
					S & $E_d$ & $M_t$ & Nb 0 pour aller jusqu'à 32bits \\
				\hline
			\end{tabular}
		\end{center}			
		
Une fois tout cela fait, on convertit le résultat en hexadécimal. Voici un exemple :		
		
		\begin{center}
			$-1027,625_{10}$ \\
			$S = 1$ \\
			$1027_{10} = 0100,0000,0011_2$ \\
			$0,625_{10} = 101_2$ \\
			$M = 0100,0000,0011,101_2$ \\
			$M_t = 0000,0000,1110,1_2$ \\
			$E_d = 1000,1001_2$ \\
			$\iff 1100,0100,1000,0000,0111,0100,0000,0000_2$ \\
			$\iff C4,80,74,00_{16}$
		\end{center} 
				
		\section{Unicode et représentation UTF}
			L'UTF-8 est rétro-compatible en fonction des chaînes de caractères à coder. Voici un tableau avec le nombre d'octets nécessaires en fonction du point de code de chaque caractère : 
		
			\begin{enumerate}
				\item Les 127 premiers caractères de l'ASCII 7 bits ont les mêmes valeurs en UTF-8 et sont donc codés sur un octet.
				\item Pour coder les caractères de valeurs comprises entre 128 et 2047, on utilise deux octets.
				\item Puis trois octets pour coder les caractères de valeurs comprises entre 2048 et 65535.
				\item Enfin, on utilise quatre octets pour les caractères de valeurs supérieures à 65535.
			\end{enumerate}
			
		\section{L'algèbre de booléens}
			Soit deux variables booléennes $a$ et $b$. On note les opérations booléennes ainsi : $a + b$, $a \cdot b$, et $\bar{a}$. Voici leurs tables de vérité :
		
			\begin{center}
				\begin{tabular}{ | c | c | c | c | c | c | }
					\hline
						$a$ & $b$ & $a + b$ & $a \cdot b$ & $\bar{a}$ & $\bar{b}$  \\
					\hline
						0 & 0 & 0 & 0 & 1 & 1 \\
					\hline					
						0 & 1 & 1 & 0 & 1 & 0 \\
					\hline					
						1 & 0 & 1 & 0 & 0 & 1 \\
					\hline
						1 & 1 & 1 & 1 & 0 & 0 \\						
					\hline
				\end{tabular}
			\end{center}
			
			Ainsi, on définit une fonction booléenne $f_1(a, b) = a_0 b_0 + \ldots + a_n b_n$.
			
			Voici les règles de calcul pour les expressions algébriques de booléens :
			
			\begin{center}
				\begin{tabular}{ | c | c | c | c | c | }
					\hline
						Loi &  & Forme + & & Forme $\cdot$  \\
					\hline
						Élément neutre & R1 & $x + 0 = x$ & R2 & $x \cdot 1 = x$ \\
					\hline
						D'idempotence & R3 & $x + x = x$ & R4 & $x \cdot x = x$  \\
					\hline
						D'inversion & R5 & $x + \bar{x} = 1$ & R6 & $x \cdot \bar{x} = 0$ \\
					\hline
						D'absorption & R7 & $x + x \cdot y = x$ & R8 & $x \cdot (x + y) = x$ \\
					\hline
						De Morgan & R9 & $\overline{x + y} = \bar{x} \cdot \bar{y}$ & R10 & $\overline{x \cdot y} = \bar{x} + \bar{y}$ \\
					\hline
						De commutativité & R11 & $x + y = y + x$ & R12 & $x \cdot y = y \cdot x$ \\
					\hline
						D'associativité & R13 & $x + (y + z) = (x + y) + z$ & R14 & $x \cdot (y \cdot z) = (x \cdot y) \cdot z$\\
					\hline
						De distributivité & R15 & $x \cdot (y + z) = x \cdot y + x \cdot z$ & R16 & $x + y \cdot z = (x + y) \cdot (x + z)$\\
					\hline
				\end{tabular}
			\end{center}
			
\end{document}
